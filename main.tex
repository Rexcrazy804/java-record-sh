% arara: pdflatex: { shell: true }
% arara: xelatex: { synctex: 1, shell: yes }
% arara: lualatex: { shell: true, interaction: nonstopmode }
\documentclass{article}	
\usepackage[a4paper, margin=1in]{geometry}
\usepackage{minted} \usepackage{subfiles}
\usemintedstyle{tango} % Optional: Set a code style (vs is just an example)
\usepackage{hyperref}
\usepackage{graphicx}
\usepackage{pdfpages}
\usepackage[labelformat=empty]{caption}

\title{%
  INCOMPLETE DOCUMENT \\
  \large code and title for 11 to 24 accurate, \\
  rest INCOMPLETE including the aim \\
}
\author{Rexiel Scarlet}
\date{2-10-2024}

\begin{document}

\maketitle
\newpage

\tableofcontents
\newpage

\section{Fibonacci Series}
\subfile{sections/prog1}
\newpage

\section{Operations On Complex Numbers}
\subfile{sections/prog2}
\newpage

\section{Package Implementation}
\section{Class Implementation}
\section{Abstract Class Implementation}
\section{Inheritance In Java}
\section{Abstract Class With Inheritance}
\section{Interface Implementation}
\section{Multithreading In Java}
\section{Exception Handling}
\newpage

\section{Validation using Swing}
\subfile{sections/prog11}
\newpage

\section{Text Field in Swing}
\subfile{sections/prog12}
\newpage

\section{Traffic Light Simulation}
\subfile{sections/prog13}
\newpage

\section{Applet In Java}
\subfile{sections/prog14}
\newpage

\section{Savings Account}
\subfile{sections/prog15}
\newpage

\section{Integer Divisions}
\subfile{sections/prog16}
\newpage

\section{Simple Interest}
\subfile{sections/prog17}
\newpage

\section{Mouse Coordinates}
\subfile{sections/prog18}
\newpage

\section{Simple Banner}
\subfile{sections/prog19}
\newpage

\section{Grid Layout Manager}
\subfile{sections/prog20}
\newpage

\section{Priority Threads}
\subfile{sections/prog21}
\newpage

\section{Employee Details}
\subfile{sections/prog22}
\newpage

\section{Update In JDBC}
\subfile{sections/prog23}
\newpage

\section{Product Details}
\subfile{sections/prog24}
\newpage

%
%\section{Time Table}
%\subsection{Aim}
%\begin{itemize}
%	\item Design the class time table in HTML using table tag
%\end{itemize}
%
%\subsection{Code}
%\inputminted[frame=lines, breaklines, breakanywhere, numberblanklines=false]{java}{./prog_2/index.html}
%
%\newpage
%\subsection{Output}
%\begin{figure}[h!]
%	\centering
%	\includegraphics[width=0.8\textwidth]{./Assets/p0201.png}
%\end{figure}
%\newpage

\end{document}
